\chapter{Grundlagen und Methoden}
Im \textbf{Kapitel 2} werden die \textbf{Grundlagen und Methoden} geklärt.


\section{Analyse des vorhandenen Systems}
Text

\subsection{Begriffe}
Text

\subsubsection{Echtzeitdaten}
Text

\subsubsection{Energietechnologie}
Text

\subsubsection{Energiesystem}
Text

\subsubsection{Front-End}
Text

\subsubsection{Back-End}
Text








\section{Anforderungen an das Produkt}
Text

\subsection{Schutz von vertraulichen Informationen}
Text

\subsection{Statistische Auswertung}
Text



\section{Architektur des Zielsystems}
Text

\subsection{Endgeräte}
Text

\subsection{Serverseitig}
Text

\subsection{Clientseitige Interaktion des Benutzers}
Text

\subsection{Framework}
Text

\subsubsection{Laravel}
\subsubsection{Angular}
\subsubsection{ASP.NET}
\subsubsection{React}
\subsubsection{Entscheidung des Frameworks}


\subsection{Front-End Templates}
\subsubsection{Bootstrap}
\subsubsection{Tailwind.css}
\subsubsection{Vue.js}
\subsubsection{Entscheidung des Front-End Templates}


\subsection{Verbindung der Datenbank mit Laravel}
\subsubsection{Laravel .env Datei }
\subsubsection{Migrations}
\subsubsection{Seeder / Factories}
\subsubsection{Datenübergabe in Laravel}


\section {Visuelle Darstellung der Energiesysteme und Energietechnologien }

\subsection{Kartendienste}

\subsubsection{Google Maps}
\subsubsection{OpenStreetMap}

\subsection{Geoinformationssystem }

\subsection{CSS-System}

\subsection{Auswahl des Anbieters}


\section{Berechtigungssystem Benutzer}

\subsection{Benutzerrollen}

\subsubsection{Administrator}
\subsubsection{Mitarbeiter}
\subsubsection{öffentlicher Benutzer}

\subsection{Berechtigungen in Laravel}


\section{Ui/Ux Design}

\subsection{Wireframe}

\subsection{Persona}


\section{Template Layout}

\subsection{Platzhalter Yield}

\subsection{Sections}
\subsection{Einbindung der definierten Sections}


\section{Laravel Befehle}

\subsection{Migration Befehle}
\subsection{Seeder und Factory Befehle}
\subsection{Model und Controller Befehle}
\subsection{Starten des Laravel Develop Servers}
\subsection{Befehle nach dem Git Pull} 



\chapter{Ergebnisdokumentation }
Im \textbf{Kapitel 3} wird die \textbf{Ergebnisdokumentation} geklärt.

\section{Laravel }
Text

\subsection{Installation}
Text

\subsection{Bootstrap Einbindung}
Text
\subsection{Grafana Einbindung}
Text

\subsection{MVC}
Text
\subsubsection{Model}
Text
\subsubsection{View}
Text
\subsubsection{Controller}
Text


\section{Datenbankanbindung in Laravel}
Text

\subsection{Datenbank Anmeldeinformationen}
Text

\subsection{Mail Server Konfigurationen}
Text

\subsection{Migrations}
Text


\section{Routen in Laravel}
Text

\subsection{Resource Routen}
Text

\subsection{GET Routen}
Text

\subsection{Auth Routen}
Text


\section{Datenbankdesign}
Text

\subsection{Erstellen eines neuen Schemas}
Text

\subsubsection{ER-Model}
Text

\subsubsection{Fremdschlüssel}
Text


\section{Corporate Design}
Text

\subsection{Vorschläge}
Text

\subsection{Änderungsvorschläge}
Text

\subsection{Finales Design}
Text

\subsection{Definierte Farben}
Text

\subsection{Überschriften}
Text

\subsection{Interaktionsfarben}
Text

\subsection{Schriftarten}
Text

\subsection{Schriftgrade}
Text

\subsection{Logo}
Text

\subsection{Verwendete Icons und deren Bedeutungen}
Text

\subsection{Map Icons}
Text

\subsection{Icons in Formularen}
Text

\subsection{Icons im DataTable}
Text

\subsection{Buttons}
Text

\subsection{Tabelle mit generellen Informationen über einzelne HTML Elemente}
Text

\subsection{Datenformate}
Text



\section{Weboberfläche}
Text

\subsection{Backend}
Text
\subsubsection{Energiesystem Erstellen}
Text

\subsubsection{Energiesystem Bearbeiten}
Text

\subsubsection{Energiesystem Löschen}
Text

\subsubsection{Energietechnologie Erstellen}
Text

\subsubsection{Energietechnologie Bearbeiten}
Text

\subsubsection{Energietechnologie Löschen}
Text

\subsubsection{Benutzerverwaltung}
Text

\subsubsection{Adresssuche}
Text


\subsection{Front-End}
Text

\subsubsection{Home}
Text

\subsubsection{Energiesysteme}
Text

\subsubsection{Galerie}
Text

\subsubsection{Impressum}
Text

\subsubsection{Datenschutz}
Text

\subsubsection{Registrierungsseite}
Text

\subsection{Login}
Text

\subsection{Registrierung}
Text

\subsection{Kartendienst Funktionalitäten}
Text

\subsubsection{Auswählen eines Energiesystems}
Text
\subsubsection{Abwählen eines Energiesystems}
Text

\subsection{Anzeige von Energiesystemen und Energietechnologien auf der Karte}
Text
\subsubsection{Energiesysteme Marker auf der Karte platzieren}
Text
\subsubsection{Energietechnologien Marker auf der Karte platzieren}
Text

\subsection{Layoutvorlage der Website}
Text



\section{DataTable}
Text
\subsection{Individueller DataTable}
Text
\subsection{Sortierfunktion}
Text
\subsection{Suchfunktion}
Text
\subsection{Seitenanzahl}
Text
\subsection{Icons}
Text
\subsection{MoveToMarker}
Text


\section{Galerie Funktionen}
Text
\subsection{Auswahl eines Energiesystems}
Text
\subsection{Energietechnologien des Energiesystems anzeigen}
Text



\section{Grafana}
Text

\subsection{Automatisches Erstellen der Dashboards}
Text

\subsection{Automatisches Erstellen der Panels}
Text

\subsection{Energietechnologien Statistiken anzeigen}
Text



\section{Einbindung von Google Maps}
Text

\subsection{Google Cloud}
Text

\subsection{Google Cloud Platform Account erstellen}
Text

\subsection{Apis aktivieren und einbinden}
Text

\subsection{Individuelle Map erstellen und einbinden}
Text



\chapter{Resümee und Ausblick}
Text


\chapter{Quellen und Literatur}
Text

\chapter{Abbildungsverzeichnis}
Text
\chapter{Tabellenverzeichnis}
Text
\chapter{Codeverzeichnis}
Text


\chapter{Begleitprotokoll gem. § 9 Abs. 2 PrO-BHS}
Text
\section{Begleitprotokoll David Pöchacker}
Text
\section{Begleitprotokoll Marcel Entner}
Text
\section{Begleitprotokoll Tobias Kronsteiner}
Text



\chapter{Anhang}
Text

\section{Verfasser der Kapitel}
Text
\subsection{David Pöchacker}
Text
\subsection{Marcel Entner}
Text
\subsection{Tobias Kronsteiner}
Text


\section{Verwendete Software}
Text
\subsection{Visual Studio Code}
Text
\subsection{Apache WebServer}
Text
\subsection{Composer}
Text
\subsection{Windows Eingabeaufforderung (CMD)}
Text
\subsection{Github VCS und Github Desktop GUI}
Text
\subsection{ phpMyAdmin}
Text
\subsection{Adobe XD}
Text
\subsection{Adobe Photoshop}
Text



\section{Projektplanung}
Text
\subsection{Projektkommunikation}
Text
\subsection{Projektstrukturplan}
Text
\subsection{Verantwortungsmatrix und Aufwandsschätzung}
Text
\subsection{Meilensteinplan}
Text
\subsection{Terminplan}
Text


\section{Inhalt von GitHub}
Text






