\chapter{Grundlagen und Methoden}
Im \textbf{Kapitel 2} werden die \textbf{Grundlagen und Methoden} geklärt.


\section{Analyse des vorhandenen Systems}
Text

\subsection{Begriffe}
Text

\subsubsection{Echtzeitdaten}
Text

\subsubsection{Energietechnologie}
Text
\subsubsection{Energiesystem}
Text
\subsubsection{Front-End}
Text
\subsubsection{Back-End}
Text








\section{Anforderungen an das Produkt}
Text

\subsection{Schutz von vertraulichen Informationen}
Text

\subsection{Statistische Auswertung}
Text
\section{Architektur des Zielsystems}
Text

\subsection{Endgeräte}
Text
\subsection{Serverseitig}
Text
\subsection{Clientseitige Interaktion des Benutzers}
Text
\subsection{Framework}
Text
\subsubsection{Laravel}
\subsubsection{Angular}
\subsubsection{ASP.NET}
\subsubsection{React}
\subsubsection{Entscheidung des Frameworks}
\subsection{Front-End Templates}
\subsubsection{Bootstrap}
\subsubsection{Tailwind.css}
\subsubsection{Vue.js}
\subsubsection{Entscheidung des Front-End Templates}
\subsection{Verbindung der Datenbank mit Laravel}
\subsubsection{Laravel .env Datei }
\subsubsection{Migrations}
\subsubsection{Seeder / Factories}
\subsubsection{Datenübergabe in Laravel}

\section {Visuelle Darstellung der Energiesysteme und Energietechnologien }
\subsection{Kartendienste}
\subsubsection{Google Maps}
\subsubsection{OpenStreetMap}
\subsection{Geoinformationssystem }
\subsection{CSS-System}
\subsection{Auswahl des Anbieters}

\section{Berechtigungssystem Benutzer}
\subsection{Benutzerrollen}
\subsubsection{Administrator}
\subsubsection{Mitarbeiter}
\subsubsection{öffentlicher Benutzer}
\subsection{Berechtigungen in Laravel}

\section{Ui/Ux Design}
\subsection{Wireframe}
\subsection{Persona}

\section{Template Layout}
\subsection{Platzhalter Yield}
\subsection{Sections}
\subsection{Einbindung der definierten Sections}

\section{Laravel Befehle}
\subsection{Migration Befehle}
\subsection{Seeder und Factory Befehle}
\subsection{Model und Controller Befehle}
\subsection{Starten des Laravel Develop Servers}
\subsection{Befehle nach dem Git Pull} 



\chapter{Ergebnisdokumentation }
Im \textbf{Kapitel 3} wird die \textbf{Ergebnisdokumentation} geklärt.

\section{Laravel }
Text
\subsection{Installation}
Text
\subsection{Bootstrap Einbindung}
Text
\subsection{Grafana Einbindung}
Text
\subsection{MVC}
Text
\subsubsection{Model}
Text
\subsubsection{View}
Text
\subsubsection{Controller}
Text

\section{Datenbankanbindung in Laravel}
Text
\subsection{Laravel .env File}
Text
\subsection{Migrations}
Text
\subsection{Datenübergabe über den Frontend Controller}
Text

\section{Routen in Laravel}
Text
\subsection{Resource Routen}
Text
\subsection{GET Routen}
Text
\subsubsection{Store}
Text
\subsubsection{Edit}
Text
\subsubsection{Destroy}
Text

\subsection{Auth Routen}
Text

\section{Datenbank Design}
Text
\subsection{Neue Schema}
Text
\subsubsection{ER-Model}
Text
\subsubsection{Fremdschlüssel}
Text
\subsubsection{Datenkatalog}
Text

\section{Weboberfläche}
Text
\subsection{Backend}
Text
\subsubsection{Verwaltung Energiesysteme / Energietechnologien}
Text
\subsubsection{Benutzerverwaltung}
Text
\subsubsection{Adresssuche}
Text

\subsection{Frontend}
Text
\subsubsection{Home}
Text
\subsubsection{Energiesysteme}
Text
\subsubsection{Bildergalerie}
Text
\subsubsection{Impressum}
Text
\subsubsection{DSGVO}
Text

\subsection{Login}
Text
\subsection{Registrierung}
Text
\subsection{Template- Layout}
Text
\subsection{Kartendienst Funktionalitäten}
Text
\subsubsection{Hinzufügen von Energiesystemen}
Text
\subsubsection{Hinzufügen von Energietechnologien}
Text
\subsubsection{Auswählen eines Energiesystems}
Text
\subsubsection{Abwählen eines Energiesystems}
Text
\subsubsection{Anzeige von Energiesystemen / Energietechnologien}
Text


\section{DataTable}
Text
\subsection{Sortierfunktion}
Text
\subsection{Suchfunktion}
Text
\subsection{Seitenanzahl}
Text
\subsection{Icons}
Text
\subsubsection{Löschen von ES/ET}
Text
\subsubsection{Editieren von ES/ET}
Text
\subsubsection{Erweiterte Ansicht der Kennzahlen}
Text
\subsubsection{Grafana-Statistiken des ausgewählten Systems anzeigen}
Text


\section{Galerie Funktionen}
Text
\subsection{Auswahl eines Energiesystems}
Text
\subsection{Energietechnologien des Energiesystems anzeigen}
Text


\section{Grafana}
Text
\subsection{Automatisches Erstellen der Dashboards}
Text
\subsection{Automatisches Erstellen der Panels}
Text
\subsection{Energiesystem Statistiken erstellen}
Text
\subsection{Energietechnologien Statistiken erstellen}
Text
\subsection{Einbinden der Statistiken}
Text

\section{Einbindung von Google Maps}
Text
\subsection{Google Cloud Platform Account erstellen}
Text
\subsection{Aktivieren der Google Maps API’s}
Text
\subsection{Einbinden des APi Keys}
Text
\subsection{API Keys erstellen}
Text
\subsubsection{Map Funktionen}
Text
\subsubsection{Map}
Text
\subsection{Eigene Map erstellen}
Text
\subsection{Map einbinden}
Text





\chapter{Resümee und Ausblick}
Text

\chapter{Quellen und Literatur}
Text

\chapter{Verzeichnisse}
Text

\chapter{Anhang}
Text
\section{Verfasser der Kapitel}
Text
\subsection{David Pöchacker}
Text
\subsection{Marcel Entner}
Text
\subsection{Tobias Kronsteiner}
Text
\section{Verwendete Software}
Text
\subsection{Visual Studio Code}
Text
\subsection{Apache WebServer}
Text
\subsection{Composer}
Text
\subsection{Windows Eingabeaufforderung (CMD)}
Text
\subsection{Github VCS und Github Desktop GUI}
Text
\subsection{ phpMyAdmin}
Text



\chapter{Projektplanung}
Text

\chapter{Inhalt von GitHub}
Text






