\chapter{Grundlagen und Methoden}
Im \textbf{Kapitel 2} werden die \textbf{Grundlagen und Methoden} geklärt.


\section{Analyse des vorhandenen Systems}
Text

\subsection{Begriffserklärung}
Text

\subsubsection{Echtzeit}
Text

\subsubsection{Energiesysteme}
Text

\subsubsection{Energietechnologie}
Text

\subsection{Vorhandenes System analysieren}
Text



\section{Anforderungen an das Zielsystem}
Text

\subsection{Verschiedene Zugriffsrechte auf das System}
Text

\subsection{Verteilung der Verwaltung von Energiesystemen und Energietechnologien}
Text

\subsection{Visuelle Darstellung der Energiesysteme / Energietechnologien auf einer Landkarte}
Text

\subsection{Statistische Auswertung}
Text

\subsection{Corporate Design}
\subsubsection{Wer benötigt ein Corporate Design Handbuch}
\subsubsection{Webseitenoberfläche}
Das Produkt gliedert sich in folgende Unterseiten:
\begin{itemize}
	\item Homepage
	\item Energiesysteme
	\item Galerie
	\item Registerpage
	\item DSGVO
	\item Impressum
\end{itemize}

 Die oben genannten Unterseiten des Produkts werden, um ein besseres Verständnis bereit zu stellen, nachfolgend mit Bildern visuell dargestellt.
  
\subsubsection{definierte Farben}
Unsere Webseitdesign basiert auf dem Front End Framework Bootstrap. 
In diesem Framework werden alle verwendeten Farben im File „app.css“ definiert
Ausschnitt aus dem File „app.css": 

\begin{lstlisting}[
	caption={app.css},
	label=Code,
	language=octave,
	numbers=left,
	firstnumber=0,
	numberfirstline=false,
	backgroundcolor=\color{mygray},
	basicstyle=\footnotesize=15,
	keywordstyle=\color{blue}
	]
	
	 --bs-blue: #0d6efd;
	--bs-indigo: #6610f2;
	--bs-purple: #6f42c1;
	--bs-pink: #d63384;
	--bs-red: #dc3545;
	--bs-orange: #fd7e14;
	--bs-yellow: #ffc107;
	--bs-green: #1b8836;
	--bs-teal: #20c997;
	--bs-cyan: #0dcaf0;
	--bs-white: #fff;
	--bs-gray: #6c757d;
	--bs-gray-dark: #343a40;
	--bs-gray-100: #f8f9fa;
	--bs-gray-200: #e9ecef;
	--bs-gray-300: #dee2e6;
	--bs-gray-400: #ced4da;
	--bs-gray-500: #adb5bd;
	--bs-gray-600: #6c757d;
	--bs-gray-700: #495057;
	--bs-gray-800: #343a40;
	--bs-gray-900: #212529;
	--bs-primary: #1b8836;
	--bs-secondary: #6c757d;
	--bs-success: #1b8836;
	--bs-info: #0dcaf0;
	--bs-warning: #ffc107;
	--bs-danger: #dc3545;
	--bs-light: #f8f9fa;
	--bs-dark: #212529;\right) 
\end{lstlisting}

Aufgrund der Vorgabe des Auftraggebers haben wir unsere Farben aus dem CSS File der bereits vorhandenen Webseite extrahiert und diese bei der Erstellung unseres Produkts verwendet.

Folgende Farben wurden von uns extrahiert und verwendet: 


\begin{tabular}{|c|c|c|c|c|}
	\hline
	 Hex & CMYK & RGB & HSV & HSL \\
	\hline
	  \#1b8836 & 80\%, 0\%, 60\%, 47\% & 27, 136, 54 & 135, 80\%, 53\% & 135, 67\%, 32\%  \\
	\hline 
	\#f8f9fa & 1\%, 0\%, 0\%, 2\% & 247, 250, 250 & 80, 1\%, 98\% & 80, 23\%, 97\% \\
	\hline
	\#e84d3d & 0\%, 67\%, 74\%, 9\% & 232, 77, 61 & 6, 74\%, 91\% & 6, 79\%, 57\% \\
	\hline
	\#212529 & 20\%, 10\%, 0\%, 84\% & 33, 37, 41 & 210, 20\%, 16\% & 210, 11\%, 15\% \\
	\hline
	\#f1f1f1 & 19\%, 10\%, 0\%, 84\% & 241, 241, 241 & 210, 20\%, 16\% & 210, 11\%, 15\% \\
	\hline
	\#0d6efd & 95\%, 57\%, 0\%, 1\% & 13, 109, 252 & 216, 95\%, 99\% & 216, 98\%, 52\% \\
	\hline
	\#21a500 & 80\%, 0\%, 100\%, 35\% & 33, 166, 0 & 108, 100\%, 65\% & 108, 100\%, 33\% \\
	\hline
	
	
\end{tabular}


\subsubsection{Überschriften}
Alle Überschriften auf unserem Produkt sind in der Farbe „\#1b8836“ eingefärbt.Jede Überschriften der Arten: <h1>, <h2> und <h3> befinden sich immer direkt in der Mitte der Webseite. 
Unter einer Überschrift folgt immer ein Fließtext, wie in \autoref{fig:Beispiel_Fließtext} visualisiert, mit einer Textfarbe von „\#212529“. Die einzige Ausnahme sind hierbei die Überschriften, welche wir als Navigatoren, wie in \autoref{fig:Navigatoren} ersichtlich ,auf der Webseite verwenden.

\begin{figure}[h]
	\centering
	\includegraphics[height=2cm,width=17cm]{images/Navigatoren}
	\caption{Navigatoren}
	\label{fig:Navigatoren}
\end{figure}

\begin{figure}[h]
	\centering
	\includegraphics[height=13cm,width=17cm]{images/Beispiel_Fließtext}
	\caption{Beispiel Fließtext}
	\label{fig:Beispiel_Fließtext}
\end{figure}

\subsubsection{Buttons}
\subsubsection{InteraktionsFarben}
\subsubsection{Schriftart}
\subsubsection{Schriftgrad}
\subsubsection{Logo}
\subsubsection{verwendete Icons}
\subsubsection{Icons mit Funktionalitäten}
\subsubsection{Webseiten Design}





\section{Architektur des Zielsystems}
Text

\subsection{Endgeräte }
Text

\subsection{Betriebssystem }
Text

\subsection{Serverseitig }
Text

\subsection{Clientseitig }
Text

\subsection{Framework }
Text
\subsubsection{Laravel}
Text
\subsubsection{Angular}
Text
\subsubsection{ASP.NET}
Text

\subsection{Frontend Templates}
Text
\subsubsection{Bootstrap}
Text
\subsubsection{Tailwind.css}
Text
\subsubsection{Vue.js}
Text
\subsubsection{Entschiedung des Frontend Templates}
Text

\subsection{Server}
Text
\subsubsection{Betriebssystem}
Text
\subsubsection{Webserver}
Text
\subsubsection{Weitere Möglichkeiten}
Text

\subsection{Datenbanksysteme}
Text
\subsubsection{MySQL}
Text
\subsubsection{NoSQL}
Text

\subsection{Verbindung Datenbanksystem Laravel}
Text
\subsubsection{Laravel .env File }
Text
\subsubsection{Migrations}
Text
\subsubsection{Seeder/Factories}
Text
\subsubsection{Laravel - Befehle}
Text


\section{Berechtigungssystem Benutzer}
Text
\subsection{Benutzerrollen }
Text
\subsubsection{Admin}
Text
\subsubsection{Mitarbeiter}
Text
\subsubsection{öffentlicher Benutzer}
Text
\subsection{Berechtigungen in Laravel}
Text



\section{Visuelle Darstellung der Energiesysteme / Energietechnologien auf einer Landkarte}
Text
\subsection{Kartendienst}
Text
\subsubsection{Google Maps}
Text
\subsubsection{Open Street Map}
Text
\subsection{Geoinformationssystem}
Text
\subsection{CSS-System}
Text
\subsection{Auswahl des Kartendienst Anbieters}
Text



\section{Ui/Ux Design}
Text
\subsection{Vorschläge}
Text
\subsection{Änderungsvorschläge des Auftraggebers }
Text
\subsection{Finales Design}
Text
\subsection{Funktionalitäten }
Text
\subsection{Benutzerhandbuch}
Text
\subsection{Design Handbuch}
Text






\chapter{Ergebnisdokumentation }
Im \textbf{Kapitel 3} wird die \textbf{Ergebnisdokumentation} geklärt.

\section{Laravel }
Text
\subsection{Installation}
Text
\subsection{Bootstrap Einbindung}
Text
\subsection{Grafana Einbindung}
Text
\subsection{MVC}
Text
\subsubsection{Model}
Text
\subsubsection{View}
Text
\subsubsection{Controller}
Text

\section{Datenbankanbindung in Laravel}
Text
\subsection{Laravel .env File}
Text
\subsection{Migrations}
Text
\subsection{Datenübergabe über den Frontend Controller}
Text

\section{Routen in Laravel}
Text
\subsection{Resource Routen}
Text
\subsection{GET Routen}
Text
\subsubsection{Store}
Text
\subsubsection{Edit}
Text
\subsubsection{Destroy}
Text

\subsection{Auth Routen}
Text

\section{Datenbank Design}
Text
\subsection{Neue Schema}
Text
\subsubsection{ER-Model}
Text
\subsubsection{Fremdschlüssel}
Text
\subsubsection{Datenkatalog}
Text

\section{Weboberfläche}
Text
\subsection{Backend}
Text
\subsubsection{Verwaltung Energiesysteme / Energietechnologien}
Text
\subsubsection{Benutzerverwaltung}
Text
\subsubsection{Adresssuche}
Text

\subsection{Frontend}
Text
\subsubsection{Home}
Text
\subsubsection{Energiesysteme}
Text
\subsubsection{Bildergalerie}
Text
\subsubsection{Impressum}
Text
\subsubsection{DSGVO}
Text

\subsection{Login}
Text
\subsection{Registrierung}
Text
\subsection{Template- Layout}
Text
\subsection{Kartendienst Funktionalitäten}
Text
\subsubsection{Hinzufügen von Energiesystemen}
Text
\subsubsection{Hinzufügen von Energietechnologien}
Text
\subsubsection{Auswählen eines Energiesystems}
Text
\subsubsection{Abwählen eines Energiesystems}
Text
\subsubsection{Anzeige von Energiesystemen / Energietechnologien}
Text


\section{DataTable}
Text
\subsection{Sortierfunktion}
Text
\subsection{Suchfunktion}
Text
\subsection{Seitenanzahl}
Text
\subsection{Icons}
Text
\subsubsection{Löschen von ES/ET}
Text
\subsubsection{Editieren von ES/ET}
Text
\subsubsection{Erweiterte Ansicht der Kennzahlen}
Text
\subsubsection{Grafana-Statistiken des ausgewählten Systems anzeigen}
Text


\section{Galerie Funktionen}
Text
\subsection{Auswahl eines Energiesystems}
Text
\subsection{Energietechnologien des Energiesystems anzeigen}
Text


\section{Grafana}
Text
\subsection{Automatisches Erstellen der Dashboards}
Text
\subsection{Automatisches Erstellen der Panels}
Text
\subsection{Energiesystem Statistiken erstellen}
Text
\subsection{Energietechnologien Statistiken erstellen}
Text
\subsection{Einbinden der Statistiken}
Text

\section{Einbindung von Google Maps}
Text
\subsection{Google Cloud Platform Account erstellen}
Text
\subsection{Aktivieren der Google Maps API’s}
Text
\subsection{Einbinden des APi Keys}
Text
\subsection{API Keys erstellen}
Text
\subsubsection{Map Funktionen}
Text
\subsubsection{Map}
Text
\subsection{Eigene Map erstellen}
Text
\subsection{Map einbinden}
Text





\chapter{Resümee und Ausblick}
Text

\chapter{Quellen und Literatur}
Text

\chapter{Verzeichnisse}
Text

\chapter{Anhang}
Text
\section{Verfasser der Kapitel}
Text
\subsection{David Pöchacker}
Text
\subsection{Marcel Entner}
Text
\subsection{Tobias Kronsteiner}
Text
\section{Verwendete Software}
Text
\subsection{Visual Studio Code}
Text
\subsection{Apache WebServer}
Text
\subsection{Composer}
Text
\subsection{Windows Eingabeaufforderung (CMD)}
Text
\subsection{Github VCS und Github Desktop GUI}
Text
\subsection{ phpMyAdmin}
Text



\chapter{Projektplanung}
Text

\chapter{Inhalt von GitHub}
Text






